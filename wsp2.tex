\documentclass[main.tex]{subfiles}

\begin{document}
\subsubsection{ASTR 100/1 Lab 1: Introduction to Astronomical Angles and Stellarium}

\textbf{Extra-credit Moon Project}
\newline
Try to get a picture of the Moon during the daytime. You can do this from anywhere, and take two pictures one zoomed out so you can see the Moon and the horizon, and the other zoomed in on the Moon. You will need to submit three things:
\begin{enumerate}
\item A picture zoomed-out as much as possible showing the horizon and the Moon. (The Moon will look very small in this picture.)
\item A second picture of the Moon zoomed-in as much as possible with your camera.
\item A screen-shot of Stellarium set up to the same location and time as when you took your picture showing the Moon's information at that time.
\end{enumerate}
One of the goals of this project is to learn \textit{\textbf{when}} you can see the Moon during the daytime, and \textbf{\textit{where}} to look. This changes according to the phase of the Moon. You will also have to plan around the weather. Stellarium can be very helpful in figuring this out. You can also search on the internet for moonrise and moonset times, and then you need to think about where the Moon will be as it goes from east to west.  


\end{document}