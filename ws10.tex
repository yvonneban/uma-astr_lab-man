\documentclass[main.tex]{subfiles}

\begin{document}
\subsubsection{Building a Constellation}
Working in teams at your table, your mission is to construct a 3-D model of your assigned constellation. 
\begin{enumerate}
\item Using Stellarium as reference, sketch your constellation on the white board, marking out the bright stars and constellation lines (lines connecting them like a stick figure). Try to make it fill up as much of the whiteboard as possible. (To show constellation lines in Stellarium, press C.)
\item In Stellarium, click on your assigned stars. Write the Name, Spectral Type, and Parallax of the star next to it on the whiteboard, and also on the table worksheet.
\item Keep the tip of the metal pole (with attached strings) about 6 ft/2 metres from the whiteboard, and at a height of about eye level or centred to your whiteboard sketch. Gently stretch a string to each assigned star and tape the end to the board, making sure that the strings are straight and untangled.
\item Calculate the distance of your assigned star in parsecs using the following distance formula:
\begin{equation*}
\text{distance (in parsecs)} = 1/\text{parallax (in arc seconds)} = \rule{2cm}{.15mm}
\end{equation*}
Copy this onto the table worksheet.
\item Pick a coloured ball that represents your assigned star according to the following:
\begin{table}[h!]
%\caption{default}
\begin{center}
\begin{tabular}{|c|c|c|c|c|c|c|}\hline
Type: & O/B & A & F & G & K & M \\\hline
Colour: & Violet & Blue & Green & Yellow & Orange & Red \\\hline
\end{tabular}
\end{center}
\label{tab:starcol}
\end{table}
\vspace{-35pt}
\begin{table}[h!]
%\caption{default}
\begin{center}
\begin{tabular}{|c|c|c|c|}\hline
Luminosity class: & I & II/III& IV/V \\\hline
Size: & Large & Medium & Small \\\hline
\end{tabular}
\end{center}
\label{tab:starsize}
\end{table}
\vspace{-20pt}

Write down the colour and size of the ball you picked on the table worksheet.

If your star has an unusual classification, try looking it up on Wikipedia for more information, and choose the closest representation you can for it.
\item Take the tip of the metal pole (where all the strings are tied together) to represent the position of the Sun. Taking 1 centimetre to represent 1 parsec, measure out distance of your assigned star from the ``Sun" and clip in place your ``star". If the calculated distance is larger than the length of the string, clip your ``star" right at the board and write its distance beside it.
\item \textbf{When everything is in place, call over an instructor to check your work.}
\end{enumerate}

\subsubsection{Changing Perspective}
\begin{enumerate}
\item Examine the stars in your constellation from the position of the ``Sun". Their pattern should look very similar to the constellation you see in the sky.

You can also go around the room to look at the other constellations.
\item On the scale of the model (1 cm = 1 parsec), how far apart are your eyes?

\rule{5cm}{.15mm}

Move 10 or 20 ``parsecs" to either side of the ``Sun" and sketch the pattern of stars from the new position. How does it compare to the original pattern?
\begin{table}[h!]
%\caption{default}
\begin{center}
\begin{tabular}{|p{0.4\textwidth}|p{0.4\textwidth}|}\hline
Original & \rule{1cm}{.15mm} pc away \\
&\\
&\\
&\\
&\\
&\\
&\\
&\\
&\\
&\\
&\\\hline
\end{tabular}
\end{center}
\label{tab:constell}
\end{table}

Pick an example in your group to copy onto the table worksheet.
\item Discuss and answer on the table worksheet: Which star appears to change position most relative to its original position as you move to the side?
\item Discuss: On the scale of your model, how far away from the Sun is Neptune? Should the constellations change when viewed from Neptune?
\end{enumerate}

\subsubsection{Moodle Lab Quiz}
Go to the section for this lab on the Moodle page and complete the End-of-Lab Quiz. Write your name on the Table Worksheet and hand it in.
\end{document}