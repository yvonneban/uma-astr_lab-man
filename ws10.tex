\documentclass[main.tex]{subfiles}

\begin{document}
\subsubsection{Building a Constellation}
Each table will construct a 3-dimensional model of a constellation. You will be assigned one of the constellations: Aquila, Bootes, Capricornus, Centaurus, Gemini, Hercules, Leo, Pegasus, Taurus, Virgo. You should select at least as many of the brightest stars in the constellation as you have people at your table. Try to pick stars that let you make out the shape of the constellation. 
\begin{enumerate} [a.]
    \item Find your constellation in Stellarium. Then sketch it oriented so that it fills most of the white board. Copy the bright stars of the constellation onto the whiteboard with lines connecting them as in the stick figure version in \textit{Stellarium}. (Tap \textbf{C} for constellation lines.)
    \item Write the Name, Spectral Type, and Parallax next to each of the stars that you are using on the whiteboard. You can find this information by clicking on each star in \textit{Stellarium}. Add this information to your table worksheet. 
    \item The table will get a bundle of strings, tape, and clips. While holding the tied end of the bundle of strings about 2 meters from the board at about eye level, tape the free end of each piece of string onto the whiteboard at the position of each star. Adjust the strings so that a gentle pull on the bundle keeps all the strings fairly straight and untangled.
    \item Each person should calculate the distance of at least one star in parsecs according the the distance formula:
    \newline
    \newline
\textit{distance (in parsecs)=1/parallax (in arc seconds)=} \rule{2cm}{.15mm}
\newline
\newline
also enter this value on the table worksheet.
\item Each person should select a colored ball for the star whose distance they calculated according to the following scheme:
\newline
\newline
(i) Type: O or B-violet; A-blue; F-green; G-yellow; K-orange; M-red.
\newline
\newline
(ii) Luminosity class: I-large; II or III-medium; IV or V-small
\newline
\newline 
If your star has an unusual classification try googling it for more information and choose the closest representation for it that you can. Write down the color and size of the ball you decide best represents your star on the table worksheet.
    \item From the point where all the strings are tied together, which represents the position of the Sun, measure out a distance of 1 centimeter per parsec distance of each star and fasten the correct color and size ball in place. If the calculated distance is larger than the length of the string, put the ball at the board and write its distance beside it.
    \item \textbf{When everything is in place, call over an instructor to check your work.}
\end{enumerate}

\subsubsection{Changing Perspective}
Examine the stars in your constellation from the position of the Sun. Their pattern should look very similar to the constellation you see in the sky. Go around the room to look at the other constellations. 
\begin{enumerate} [a.]
    \item On the scale of the model, about how far apart are your eyes?\rule{2cm}{.15mm} 
    \newline 
Move 10 or 20 parsecs to either side of the Sun and sketch the pattern of stars from the new position. How does it compare to the original pattern?
\newline
Copy one of your group's better examples to the table worksheet.
\item Which star appears to change position most relative to its original position as you move to the side? \textbf{\textit{Discuss this at your table and answer on the table worksheet.}}
\item On the scale of your model, how far away from the Sun is Neptune, roughly speaking? Should star pattern change at all? \textbf{\textit{Discuss.}}
\end{enumerate}

\subsubsection{Moodle Lab Quiz}
Go to the section for this lab on the Moodle page and complete the End-of-Lab Quiz. Write your name on the Table Worksheet and hand it in.

\textbf{Extra-credit Moon Project}
\newline
Try to get a picture of the Moon during the daytime. You can do this from anywhere, and take two pictures one zoomed out so you can see the Moon and the horizon, and the other zoomed in on the Moon. You will need to submit three things:
\begin{enumerate} [a.]
    \item A picture zoomed-out as much as possible showing the horizon and the Moon. (The Moon will look very small in this picture.)
    \item A second picture of the Moon zoomed-in as much as possible with your camera. 
    \item A screen-shot of Stellarium set up to the same location and time as when you took your picture showing the Moon's information at that time.
\end{enumerate}
One of the goals of this project is to learn \textit{\textbf{when}} you can see the Moon during the daytime, and \textbf{\textit{where}} to look. This changes according to the phase of the Moon. You will also have to plan around the weather. Stellarium can be very helpful in figuring this out. You can also search on the internet for moonrise and moonset times, and then you need to think about where the Moon will be as it goes from east to west.  

\end{document}