\documentclass[main.tex]{subfiles}

\begin{document}
\subsubsection{Exploring the ecliptic}
\begin{enumerate}
\item Open Stellarium and set it to your home location. You can also stop time (the pause button in the bottom panel) and set the time to noon.
\item Make the constellations visible.
	\begin{enumerate}
	\item Turn off the Atmosphere visualisation.
	\item Open the Sky and Viewing options window.
	\item Go to the Starlore tab.
	\item Under Options, check Show labels and Show boundaries.
	\end{enumerate}
\end{enumerate}
\begin{enumerate}
\item Find out what constellation the
	\begin{enumerate}
	\item Sun
	\item Moon
	\end{enumerate}
are in today.
\item Click on the Sun and hit the spacebar. This will centre and lock on the Sun.
\item Move forward in time by jumping by day, by holding down the = key.
\item Find out the constellations that the Sun moves through over a year.
\end{enumerate}
What constellations does the Sun move through? What is causing the Sun to move through the stars? What other things change position among the stars as time passes?

\subsubsection{What's your sign?}
\begin{table}[htbp]
\caption{Horoscope signs and dates}
\begin{center}
\begin{tabular}{|p{1cm}|p{1cm}|p{1cm}|p{1cm}|p{1cm}|p{1cm}|p{1cm}|p{1cm}|p{1cm}|p{1cm}|p{1cm}|p{1cm}|p{1cm}|}\hline
\textbf{Sign} & Aries & Tau-rus & Gemi-ni & Can-cer & Leo & Virgo & Libra & Scor-pio & Sagi-ttarius & Capri-corn & Aqua-rius& Pisces \\\hline
\textbf{Start} & 21 Mar & 20 Apr & 21 May & 21 Jun & 23 Jul & 23 Aug & 23 Sep & 23 Oct & 22 Nov & 22 Dec & 20 Jan & 19 Feb \\\hline
\textbf{End} & 19 Apr & 20 May & 20 Jun & 22 Jul & 22 Aug & 22 Sep & 22 Oct & 21 Nov & 21 Dec & 19 Jan & 18 Feb & 20 Mar\\\hline
\end{tabular}
\end{center}
\label{tab:horo}
\end{table}

\begin{enumerate}
\item Shift Stellarium to your birth date. What constellation was the Sun in? Save a screenshot of the Sun's location.

According to the ancient idea of astrology, this is known as your sun sign or horoscope sign, which somehow determines your characteristics.
\item The dates for each horoscope sign according to modern astrologers is given in the table. What is your sign based on the table?
\item Based on Stellarium, over what range of dates is the Sun actually in the constellation corresponding to your horoscope sign?
\item Set Stellarium to a date in the middle of your horoscope sign. Jump back in time in 500-year blocks until the Sun is actually in the middle of your horoscope sign. How far did you have to go?

N.B.: To enter a B.C. date, you may need to enter the number first, then put ``-" in front.
\end{enumerate}

\subsubsection{Precession of Earth's Axis}
In fact, the position of the Sun on a particular date shifts steadily through the sky because of the precession of the Earth's axis. Let's visualise this in Stellarium.
\begin{enumerate}
\item Jump back to today and turn on the equatorial (NOT azimuthal!) grid.
\item Look at the North celestial pole. Which star is nearest to it? Which star was nearest to it when horoscopes were aligned with the zodiacal constellations?
\item Find the star Vega. Jump back in time in 1000-year blocks. When was Vega the North star?
\item Jump forward in time in 1000-year blocks. When will Vega be the North star again?
\end{enumerate}

\subsubsection{Constellations in Other Cultures}
Different cultures have come up with different groupings of stars and different stories associated with them. Let's explore them!
\begin{enumerate}
\item Go back to your birth date.
\item Pick constellations from another culture.
	\begin{enumerate}
	\item Go to Sky and viewing options and go to the Starlore tab.
	\item Choose another cultural constellation system (other than Western). Which constellation is the Sun in?
	\end{enumerate}
\end{enumerate}
If no labelled constellation is available, do some research online and tell us about what you find.

On Moodle, write about the culture you chose and what you've learnt about the constellation.

\subsubsection{The Motion of Mars}
The planets move through the stars of the zodiac and show some surprising motions, like Mars.
\begin{enumerate}
\item Centre your view on Pisces.
	\begin{enumerate}
	\item Set Stellarium to 1 June 2020.
	\item Turn off the atmosphere and ground visualisations.
	\item Switch to the equatorial axis. (This is NOT the same as turning on the equatorial grid!)
	\item In Sky and viewing options, turn on constellation lines and labels as before.
	\item Centre and zoom in on Pisces.
	\end{enumerate}
\item Find Mars on the sky.
\item Plot the position of Mars every month for one year (i.e. once a month from June 2020 to June 2021).
\item Connect the points. How many days was it moving backwards (retrograde)?
\item Go back through the same period. This time, zoom in on Mars and take note of its magnitude, phase, and size.
\item Note the highest and lowest magnitudes and note them on the chart.

Note that the lower or more negative the magnitude, the brighter an object is. The magnitude scale is logarithmic: 5 magnitudes lower is 100$\times$ brighter.
\item Repeat for all the planets. Do they all go retrograde (i.e. move backwards for some time)?
\end{enumerate}

\subsubsection{Moodle Lab Quiz}
Go to the section for this lab on the Moodle page and complete the End-of-Lab Quiz.

\end{document}