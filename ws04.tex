\documentclass[main.tex]{subfiles}

\begin{document}
\subsubsection{Actual Scale of the Earth and the Moon (Class demo)}
\begin{enumerate}
\item Suppose the Moon is the size of a golf ball. On this scale, how big is the Earth? \textbf{Lower} your hand when you think the demo balloon is at the right size.

\item On this scale, how far would the golf ball Moon be from Earth? \textbf{Lower} your hand when you think the demo balloon is at the right distance.

Relative to the size of the golf ball, this distance is the same as the as the distance from the Earth to the Moon, relative to the size of the real Moon. Hence, \textbf{the angular sizes are the same.} If you take a zoomed-in picture of the golf-ball at this distance and compare it to your zoomed-in calibration grid, you will find the golf ball has the same angular size as the real Moon.

\item As part of the extra credit project, take a picture of the golf-ball Moon from the to-scale distance.
\end{enumerate}

\subsubsection{Modelling the Moon's Phases}
You will each get a golf ball with a hole in it so you can put it on a pen or pencil and hold it up. Each table will have a light source that will represent the Sun.

\begin{enumerate}
\item To first understand Earth's rotation in relation to the Sun, begin by picturing your head as the Earth. Imagine the top of your head is the North Pole, with Boston at your left eye, and San Francisco your right eye. Take the light bulb on the table to be the Sun.
	\begin{enumerate}[a.]
	\item Which way does your head face when it is noon in Boston?
	\item Which way does your head face when it is noon in San Francisco?
	\item Which way does your head turn to go from noontime in Boston to noontime in San Francisco?
	\end{enumerate}

\item Let's examine the Moon's phases as it orbits the Earth. Make sure that you can see the portion of the Moon lit by the ``Sun" in your table.
	\begin{enumerate}[a.]
	\item Look for the \textbf{crescent} phase, and estimate the angle between the Sun and the Moon (with the Earth at the vertex) when the Moon is a crescent.
	\item Can you ever see a crescent Moon at midnight?
	\item Where is the Moon when it is \textbf{new}?
	\item Where is the Moon when it is \textbf{full}?
	\item Where is the Moon when it is \textbf{gibbous}?
	\end{enumerate}

\item Suppose the Moon is at \textbf{first quarter}.
	\begin{enumerate}[a.]
	\item When should it cross your \textbf{meridian}?
	\item When does it rise, and when does it set?
	\end{enumerate}

\end{enumerate}

\subsubsection{Observing the Moon's Phases}
\begin{enumerate}
\item In groups of 2-3, start up Stellarium.

\item Set up Stellarium for Amherst Center (F6) on 23 February 2022 (F5). Look toward the \textbf{east} and adjust the time until you see the Moon rising.
	\begin{enumerate}[a.]
	\item What time does the Moon rise?
	\item Press the semicolon ``;" key to turn on the meridian line. What time does the Moon cross the meridian?
	\item What time does it set?
	\end{enumerate}

\item If you were taking a picture of the Moon in the daytime, what is the range of times you could take a picture on 23 February 2022?

\item Zoom in on the Moon and look at its shape. 
	\begin{enumerate}[a.]
	\item What is its phase on 23 February 2022?
	\item Turn off the ground and center the view on the Moon. Advance the day by opening the Date/Time window, and then clicking the up-arrow ($\uparrow$) above the day's date, until the Moon is next at first quarter. What is the date?
	\item How can you tell whether a ``half-lit" Moon is in the first or third quarter?
	\end{enumerate}

\item On this new date,
	\begin{enumerate}[a.]
	\item What time does the Moon rise?
	\item What time does the Moon cross the meridian?
	\item What time does it set?
	\end{enumerate}

\item Advance the date by 1 day. What time does the Moon rise? Cross the meridian? Set on this date?
	\begin{enumerate}[a.]
	\item What is the Moon's phase?
	\item What time does the Moon rise?
	\item What time does the Moon cross the meridian?
	\item What time does it set?
	\item What are the differences from Part (6)?
	\end{enumerate}

\item Keeping the same time, change your location to Australia (hit \textbf{F6} and click on Australia in the location window). What phase is the Moon in?

\item Now, change your location to the Moon and look back at the Earth (search for and select ``Moon" in the Location window). What phase is the Earth in?

\end{enumerate}

\subsubsection{Lab Quiz on Moodle}
Go to the Lab 4 section on the Moodle page and complete the End-of-Lab Quiz.

%\textbf{Sunset Project:} Continue trying to get a good sunset (or sunrise) photo before October 16. Is there room along the horizon to track the Sun's changing position? Show your location photos to someone else and ask if they think they could return to the exact same location without you being there.

%Once you've gotten a good sunset picture, try to get back to take a second picture from the same spot a week or more later-that will be the next piece of the sunset project.

\end{document}