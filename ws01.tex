\documentclass[main.tex]{subfiles}

\begin{document}
\subsubsection{Background Survey on Moodle}
Log in to Moodle (on smartphone, laptop, or one of the computers at the table). Go to the section for Astronomy Lab 1 and complete the Astronomy Background Survey. Don't worry if you don't know the answers, just answer every question as best you can and you will receive full credit.

\subsubsection{Stellarium}
Stellarium is a planetarium software that helps us visualise the sky over time. We'll use it to begin our exploration of astronomy.
\begin{enumerate}
\item Go to \url{http://stellarium.org/} and download and install the correct version for your operating system. If you can't find it, use Stellarium Web.
\item Open the Location Window from the left-hand menu panel or by pressing F6. In the search bar under the top right, menu type ``Amherst" and select ``Amherst Center, United States".
	\begin{itemize}
	\item Make sure it's ``Amherst Center", not ``Amherst" (that's in NY somewhere).
	\item \textbf{Do not type} into the ``Name/City" box. That renames the location.
	\end{itemize}
\item Close the Location Window and open the Date/Time Window from the left-hand menu panel or by pressing F5. Set the time to 7:00pm tonight. Close the window.
\item Look at what the sky will look like tonight. Drag the view and scroll up or down to zoom in or out respectively.
\end{enumerate}

\subsubsection{Positions of Astronomical Objects}
In astronomy, we use angles to measure positions and separations of objects in the sky. From your point of view, an object's position is given by \textbf{azimuth} and \textbf{altitude}. Answer the following questions.
\begin{enumerate}
\item How many degrees are in a circle?
\item How many degrees are in a right angle?
\item What are longitude and latitude?
\item What is the azimuth/altitude system?
\end{enumerate}

Astronomers use the \textbf{sexagesimal} system for recording angles. Each degree is divided into 60 \textbf{arc minutes} (\si{\arcminute}), and each arc minute is divided into 60 \textbf{arc seconds} (\si{\arcsecond}). An azimuth of \SI{240}{\degree}\SI{38}{\arcminute}\SI{12}{\arcsecond} is read as 240 degrees, 38 arc minutes, and 12 arc seconds.

Set the time to 2022/1/31 at 6:00pm. Locate the object and click on it, then answer the following. At this time, what is the azimuth and altitude of: 
\begin{center}
\begin{tabular}{|l|p{5cm}|p{5cm}|}\hline
Object & Azimuth & Altitude \\\hline
Betelguese & & \\
&&\\\hline
Jupiter & & \\
&&\\\hline
Sirius & & \\
&&\\\hline
\end{tabular}
\end{center}
   
Now advance the time by a few minutes. Do the azimuths and altitudes increase or decrease?

\subsubsection{Moodle Lab Quiz}
Go to the section for this lab on the Moodle page and complete the End-of-Lab Quiz.

\textbf{If you logged into your Moodle account from a classroom computer, be sure to log back out!}

\end{document}