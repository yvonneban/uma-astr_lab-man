\documentclass[main.tex]{subfiles}

\begin{document}
%\rightline{\today}
\subsubsection{Installation}
You can download Stellarium for your own computer (Mac, Windows, Linux) at \url{http://www.stellarium.org}. The software is free, and there are links to extensive documentation on the website.

\subsubsection{Quick Usage Guide} After installing it, use the following walk-through to familiarise yourself with the basics of how the program works.

N.B.: If pressing the function keys (F1 -- F12) isn't working, press down the Fn key (keyboard lower left corner) and press the function key.
\begin{enumerate}
\item
Open Stellarium.
\item
The default location is Paris (seen in the lower left corner). For observations made from Amherst, we need to change the location.
\begin{enumerate}[i.]
    \item
    Move the cursor to the lower-left side of the window and click on ``Location Window", or press F6.
    \item
    In the ``Location" window, in the entry field next to the magnifying glass (for searching), type ``Amherst". Then, click on ``Amherst Center, United States".
    
    (\textbf{DO NOT} type ``Amherst" into Name/City. It will just rename Paris to Amherst, but won't change the location. Also \textbf{DO NOT} select ``Amherst, United States".)
    \item
    Close the window by clicking on the cross ($\times$) in the upper right corner.
\end{enumerate}
\item
Change the Date/Time to a value, e.g. 2015/09/01, 14:00:00.
\begin{enumerate}[i.]
    \item
    Move the cursor to the lower-left side of the window and click on ``Date/Time Window", or press F5.
    \item
    Change the date and time either by entering the values directly or using the arrow keys to increase or decrease the values.
    \item
    Close the window by clicking on the cross ($\times$) in the upper right corner.
\end{enumerate}
\item
Click and hold the cursor on a spot in the window, and drag it around to look in various directions.
\item
Zoom in by scrolling \textbf{UP} and zoom out by scrolling \textbf{DOWN} (on touchpad, mousepad, mouse, etc.)
\item
Zoom out till you see the full sky in your screen. You should see only the Sun.
\begin{enumerate}[i.]
    \item
    To turn on and off the atmosphere visualisation, move the cursor to the bottom of the window on the left side and click on the cloud icon, or press A.
    \item
    To turn on and off the ground visualisation, move the cursor to the bottom of the window on the left side and click on the trees icon, or press G.
    \item
    To turn on and off the constellation lines, move the cursor to the bottom of the window on the left side and click on the ``N"-shaped icon, or press C.
    \item
    To turn on and off the constellation names, move the cursor to the bottom of the window on the left side and click on the mirrored ``N"-shaped icon, or press V.
    \item
    To turn on and off the constellation art, move the cursor to the bottom of the window on the left side and click on the person-shaped icon, or press R.
    \item
    To turn on and off the stars, press S.
\end{enumerate}
\item
Click on Saturn. Notice that some information about Saturn pops up at upper left corner of the window.
\begin{enumerate}[i.]
    \item
    Press the space bar to center the view on the selected object, Saturn.
    \item
    Zoom in until the field of view (FOV at the bottom of screen) is \SI{0.1}{\degree}. What objects other than Saturn can you see?
    \item
    To start and stop the progression of time, move the cursor to the bottom of the window on the left side, move to the right end of the toolbar, and click on the ``play" icon, or press K.
    \item
    To increase the progression of time, move the cursor to the bottom of the window on the left side, move to the right end of the toolbar, and click on the ``fast forward" icon, or press L.
    \item
    To decrease the progression of time, move the cursor to the bottom of the window on the left side, move to the right end of the toolbar, and click on the ``rewind" icon, or press J.
    \item
    To return to the present time, move the cursor to the bottom of the window on the left side, move to the right end of the toolbar, and click on the ``hourglass" (2 triangles vertically stacked, points facing each other) icon, or press 8.
\end{enumerate}
\item
Search for an object, e.g. the Sun.
\begin{enumerate}[i.]
    \item
    Move the cursor to the lower-left side of the window and click on ``Search Window", or press F3.
    \item
    Enter the name of the object you are searching for.
    \item
    Close the window by clicking on the cross ($\times$) in the upper right corner.
\end{enumerate}
\item Play around and familiarise yourself with the software. For help, press F1.
\end{enumerate}
\end{document}