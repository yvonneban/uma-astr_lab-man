\documentclass[main.tex]{subfiles}

\begin{document}
\subsubsection{Class information}
Astronomy 100/1 Lab is conducted in tandem with Astronomy 100 and(/or) 101 courses and is required to be taken simultaneously.

The goal of Astronomy 100/1 Lab is to introduce some basic aspects of observational astronomy, and help relate that knowledge to the broader science of astronomy, which the lecture portion of the course focuses on. We will explore many aspects of the day and night sky to learn how objects move over the course of the day, month, and year, and how the patterns of what we see in the sky change from different parts of the Earth. We will also carry out several activities to help explain important ideas in astronomy.

Each semester, 6 independent sections are held on Monday, Wednesday, and Friday of every school week at 12:20-1:10pm (1220-1310) and 1:25-2:15pm (1325-1415). Every section is held in the Integrated Learning Center (ILC) room S220.

To reduce paper wastage, much of the course information is posted on Moodle. Check there regularly for announcements and any schedule changes as well.

\subsubsection{Instructors}
\begin{table}[h!]
%\caption{default}
\begin{center}
\begin{tabular}{p{4.15cm}p{5cm}p{5cm}}
Course coordinator:& Prof. & @astro.umass.edu\\
Mon 12:20pm lead TA:&  & @astro.umass.edu\\
Mon 1:25pm lead TA:&  & @astro.umass.edu\\
Wed 12:20pm lead TA:&  & @astro.umass.edu\\
Wed 1:25pm lead TA:&  & @astro.umass.edu\\
Fri 12:20pm lead TA:&  & @astro.umass.edu\\
Fri 1:25pm lead TA:&  & @astro.umass.edu\\
\end{tabular}
\end{center}
\label{default}
\end{table}
\vspace{-20pt}

When contacting your section TA or the course coordinator, please remember to put ``Astro Lab" in the subject of your email. Most questions can be answered by your section TA. However, also note that Astronomy 100/1 Lab is \textbf{distinct} from Astronomy 100/1 lecture, so \textbf{the lab TAs cannot help you with questions about the lecture class}.

\subsubsection{Grading}
The Astronomy Lab total grade makes up 25\% of the overall Astronomy 100/1 course grade, and does not have a separate grade. The remaining 75\% of the Astronomy 100/1 course grade comes from lecture assignments and exams.

There are 8 labs that together contribute 80\% of your overall lab score, and the remaining 20\% is from the lab Sunset Project. There are also opportunities for extra credit.

Most of the lab work is done during section, so attendance and participation is critical to your grade. Your lab grade will usually raise your overall course grade \textbf{if} you attend lab every week and take all the required photos correctly. However, failure to attend will certainly lower your overall grade.

Grading of each lab is based primarily on participation, which is recorded using Moodle. Hence, please bring a smartphone, tablet, or notebook computer to lab. If you don't have one available, you can log in on one of the computers in class.

\textbf{Submitting a photo that someone else took is plagiarism under the UMass Academic Honesty Policy, and it will be dealt with accordingly.} If you have questions about what is or isn't allowed, please ask your lab instructor.

\subsubsection{Attendance and schedule}
\begin{table}[htb]
\caption{Lab schedule}
\begin{center}
\begin{tabular}{|c|c|c|c|c|c|}\hline
\textbf{Week} & \textbf{Month} & \textbf{M} & \textbf{W} & \textbf{F} & \textbf{Lab} \\\hline
1 &  &  &  &  & 1 \\\hline
2 &  &  &  &  & 1 \\\hline
3 &  &  &  &  & 2 \\\hline
4 &  &  &  &  & 3 \\\hline
5 &  &  &  &  & 4 \\\hline
6 &  &  &  &  & Makeup \\\hline
7 &  &  &  &  & 5 \\\hline
8 &  &  &  &  & 6 \\\hline
9 &  &  &  &  & 7 \\\hline
10 &  &  &  &  & 8 \\\hline
11 &  &  &  &  & 9 \\\hline
12 &  &  &  &  & Makeup \\\hline
13 &  &  &  &  & Break \\\hline
14 &  &  &  &  &  \\\hline
15 &  &  &  &  & Reading \\\hline
16 &  &  &  &  & Exams \\\hline
\end{tabular}
\end{center}
\label{tab:sched}
\end{table}%
\vspace{-20pt}

For the project, you will need access to a digital camera (a cell phone camera is fine). You will take your own original pictures and upload them to Moodle over the course of the semester. Taking the photos is not difficult, but does require some advance planning. We will cover this in detail in Lab 3.

We also advise you to keep your lab worksheets from each week, since we will sometimes refer to them in later labs.

\textbf{If you must miss a lab at your regularly scheduled time}, e.g. if you are ill or have a University-sponsored event:

First, try to make up the lab at one of the other sessions the same week. Look at the lab schedule in Table~\ref{tab:sched} for a section with the same Lab number, and make sure to email the course coordinator with your proposed switch and reason. 

If you can't make it, there will be make-up lab sessions for each lab later in the semester, during make-up weeks, also listed in the schedule. A specific schedule will be released closer to the date. Note that make-up sessions have to accommodate the team-based nature of the labs, so individual make-ups are not possible.

\end{document}