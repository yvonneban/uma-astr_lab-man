\documentclass[main.tex]{subfiles}

\begin{document}

You will be working in groups at each table to carry out 2 experiments. For these experiments you have:
\begin{enumerate}
\item A flashlight
\item A metre stick
\item A protractor
\end{enumerate}

First, decide on your roles in the experiments, e.g., experimentalist, note taker, calculations, number checker, graphing, etc. Once you've decided as a team, write down your members and roles on the Lab 6 Table Worksheet.

\subsubsection{Power vs. Distance}
In this experiment, you will measure how distance and angle affect the amount of sunlight, and corresponding energy, that falls on the ground.

\begin{enumerate}[1.]
\item For this experiment, push the sleeve on the front part of the flashlight all the way \textbf{in}, i.e., to make the total length of the flashlight as \textbf{short} as possible.

\item Shine it on the white board from \SI{0.5}{ft}. As a group, find and agree on the position of the edge of the beam.

N.B.: it may be faint and somewhat confusing because of reflections inside the flashlight.

\item Next you will shine the flashlight on the white board and measure the area of the circle from several different distances. Make measurements of \textit{a} and \textbf{b} (should be approximately the same) with the light bulb at 0.5, 1, 1.5, and 2 meters. \textit{The idea here is the flashlight at 1 meter is analogous to sunlight at 1 AU. Earth's distance varies just slightly throughout its orbit, between 0.983 and 1.017 AU, less than a 2\% variation. Mars on the other hand is about 50\% farther from the Sun (and its distance varies by 10\%), and Venus about 30\% closer to the Sun (and its distance varies by <1\%). From your graph you can see how the intensity of sunlight varies with distance.}

\item Fill in \textit{a} and \textit{b} in the table then carry out the calculations to find the area the light is spread over. To simplify the calculations a bit, we will imagine that the flashlight produces 10,000 watts of light output. What we want you to determine is the number of watts per square centimeter at each distance. The \textit{area} of your beam equals 0.79\textit{×a×b.} Measure \textit{a} and \textit{b} in centimeters to find the area in square centimeters, and then calculate 10,000 watts/\textit{area} to find the watts per square centimeter. (Fill in the table.) 
\item Plot you results versus the distance between the board and the flashlight's light bulb. Pick one other distance to fill in a gap in your graph or to check one of the first measurements if it seems inconsistent with the others.

\end{enumerate}

\subsubsection{Power vs. Angle}
You will next repeat the previous experiment, but now you will measure the effect of changing the angle while keeping the flashlight bulb at a constant distance of 1 meter from the whiteboard. \textit{This models what happens on different parts of Earth's surface while the Sun is shining on the ground at an angle. (Mar's axis is tilted a little more than Earth's and is highly variable, while Venus's is very small, just a few degrees, but it spins backward.)}

\begin{enumerate}[a.]
\item Change your role in this experiment, and write down your new role on the table worksheet.

\item This time you will use the meter stick to keep the flashlight's light bulb at a fixed distance of 1 meter from the board, and a protractor to measure the angle. Note that you can use markers to outline the dimensions \textit{a} and \textit{b} of the ellipse created by the flashlight. Try to keep the ellipse of the beam centered on the base of the meter stick so \textit{b} stays about the same.

\item You will calculate the area of the beam for several angles. In addition to 90°, which you can copy over from the first experiment. Make measurements at 71°, 47°, and 24°. These correspond to the altitude of the Sun in Amherst at noon on the summer solstice, the equinoxes, and the winter solstice respectively.

\item As before, divide 10,000 watts by the area you measured at each angle, determine your y-axis scale, and fill in the table with your measurements and calculations and graph the data.

\item From your graph, at what angle does the intensity of light match the intensity at 1.5m (50\% farther away)? Try to estimate that angle then carry out one last measurement at that angle and fill it in the your table.

\end{enumerate}

\subsubsection{Distance vs. Angle}
The angle you measure relative to the whiteboard is like the "altitude" of the Sun. So when the flashlight is shining straight at the board, it is like having the Sun at the zenith, 90° from the horizon.\par

At noon on the winter solstice in Amherst the Sun has an altitude of about 24°. Compare your graphs to estimate the distance at which the watts per square centimeter (at 90°) is the same as the watts per square centimeter at an angle of 24°. \textbf{What distance do you estimate matches the light intensity at 24°? \rule{2cm}{.15mm} (record this for the end-of-lab quiz})

\subsubsection{Lab Quiz on Moodle}
Go to the Lab 6 section on the Moodle page and complete the End-of-Lab Quiz. Write your name on the Table Worksheet and hand it in.

%Before you leave, open the end-of-lab quiz and fill in the answers. Take a picture of the graphs in your table worksheet and make sure you wrote your name and roles on the table worksheet. Turn in the table worksheet after everyone has gotten a picture of it

\end{document}