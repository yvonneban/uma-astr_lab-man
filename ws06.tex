\documentclass[main.tex]{subfiles}

\begin{document}

As a team at your table, you will carry out 2 experiments. For these experiments you have:
\begin{enumerate}
\item A flashlight
\item A metre stick
\item A protractor
\end{enumerate}

First, decide on your roles in the experiments, e.g., experimentalist, note taker, calculations, number checker, graphing, etc. Once you've decided as a team, write down your members and roles on the Lab 6 Table Worksheet. Make sure everyone participates in observation at some point in experiment 1 or 2.

N.B.: Be sure to mark your table number on the end-of-lab quiz or you may not receive credit!

\subsubsection{Power vs. Distance}
In this experiment, you will measure how distance and angle affect the intensity of sunlight, and corresponding energy, that falls on the ground.

\begin{enumerate}[1.]
\item For this experiment, push the sleeve on the front part of the flashlight all the way \textbf{in}, i.e., make the total length of the flashlight as \textbf{short} as possible.

\item Shine it at the whiteboard straight on from 0.5 ft away. As a group, find and agree on the position of the edge of the beam.


You can use markers to outline the dimensions $a$ and $b$ of the flashlight beam. They should be about the same, i.e., the beam should be circular.

\item Next, shine the flashlight at the whiteboard and measure the height $a$ and width $b$ of the beam from several different distances. Make measurements with the light bulb at 0.5, 1, 1.5, and 2 ft.

\item Fill in $a$ and $b$ in the table. Then, carry out the calculations to find the area the light is spread over, i.e., calculate the number of watts per square centimetre at each distance.

To simplify the calculations, assume that the flashlight produces 10,000 watts (\si{\W}) of light output. Measure $a$ and $b$ in centimetres, and then use $\text{area} = 0.79\ ab$. Next, calculate 10,000 watts/area to find the intensity in watts per square centimetre.

\item Plot your results of intensity versus distance between the board and the flashlight's light bulb. Pick one other distance to fill in any gap in your graph, or to check one of the first measurements if it seems inconsistent with the others.

\end{enumerate}

\subsubsection{Power vs. Angle}
For this experiment, we will measure the effect of changing the angle while keeping the flashlight bulb at a constant distance of 1 ft from the whiteboard.

\begin{enumerate}[1.]
\item Change your role in this experiment, and write down your new role on the table worksheet.

\item For this experiment, pull the sleeve on the front part of the flashlight all the way \textbf{out}, i.e., make the total length of the flashlight as \textbf{long} as possible.

\item Use the yardstick to keep the flashlight's light bulb at a fixed distance of 1 ft from the board, and use a protractor to measure the angle.

As before, you can use markers to outline the dimensions $a$ and $b$ of the flashlight beam. It will elongate into an ellipse at smaller angles. Try to keep the ellipse of the beam centered on the base of the yardstick, so the width $b$ stays about the same.

\item Make measurements at \SI{90}{\degree}, \SI{71}{\degree}, \SI{47}{\degree}, and \SI{24}{\degree}. Assume that these smaller angles correspond to the altitude of the Sun in Amherst at noon on the summer solstice, the equinoxes, and the winter solstice respectively.

As before, use $\text{area} = 0.79\ ab$ to find the area of the flashlight beam.

\item As before, divide 10,000 watts by the area you measured at each angle. Determine your $y$-axis scale based on the range of values you obtain, and fill in the table with your measurements and calculations. Then, plot the data.

\item From your plot, at what angle does the intensity of light match the intensity at 1.5 ft (50\% farther away than 1 ft)? Try to estimate what that angle would be, then carry out one last measurement at that angle and fill it out in your table.

\end{enumerate}

\subsubsection{Distance vs. Angle}
The angle you measure relative to the whiteboard is like the ``altitude" of the Sun. So when the flashlight is shining straight at the board, it is like having the Sun at the zenith, \SI{90}{\degree} from the horizon.

At noon on the winter solstice in Amherst, the Sun has an altitude of about \SI{24}{\degree}. By comparing your graphs, estimate the distance where the light intensity (in watts per square centimetre), shining at \SI{90}{\degree}, is the same as the light intensity (in watts per square centimetre) at a distance of 1 ft and an angle of \SI{24}{\degree}.\\

What is your estimate? \rule{5cm}{.15mm} (Record this for the next section.)

\subsubsection{Moodle Lab Quiz}
Go to the section for this lab on the Moodle page and complete the End-of-Lab Quiz. Write your name on the Table Worksheet, take a picture of it, and hand it in.

%Before you leave, open the end-of-lab quiz and fill in the answers. Take a picture of the graphs in your table worksheet and make sure you wrote your name and roles on the table worksheet. Turn in the table worksheet after everyone has gotten a picture of it

\end{document}