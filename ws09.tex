\documentclass[main.tex]{subfiles}

\begin{document}
\section*{ASTR 100/1 Lab 9: The HR Diagram}
\subsection*{1. Stellar Properties}
Sitting in groups of 3 or fewer at a computer. Start up \textit{Stellarium,} and put yourself in Amherst. Search for the star \textit{Procyon,} and if necessary adjust the time until it is above the horizon, then work \textit{carefully} through the following together. 

\textit{Note:} It is easier to find the data we need in Stellarium if you limited the information it displays. Open up the configuration window (F2), then click the information tab and click "None" at the top. Select the following to display: Name, Catalog number, Visual magnitude, Absolute magnitude, Distance, and Additional information.
\begin{enumerate} [a.]
    \item What is the star's magnitude? What is its absolute magnitude? Why do these differ? 
    
    A reminder of magnitude scale is shown below.
    ADD IMAGE
    \item Fill in the first 5 columns below using the information in Stellarium:
\end{enumerate}

\subsection*{2. Stellar Sizes}
Start up a web browser, and go to the following address (or click on the link in Moodle): \textbf{http://astronomy.nmsu.edu/geas/labs/hrde/hrd\textunderscore explorer.html} Under \textbf{Options}, change the x-axis scale to "\textbf{B-V color index}" and the y-axis scale to "\textbf{magnitude}," and \textbf{check} all but "Instability Strip" box. This shows an interactive "Hertzsprung-Russell Diagram" that helps us to examine the properties of stars.
\begin{enumerate} [a.]
    \item \textbf{Adjust the Temperature slider} so that the red arrow on the bottom axis of the graph matches the star's B-V color value. Enter the star's temperature in the table.
    \item \textbf{Adjust the Luminosity Slider} so the red arrow on the left axis of the graph matches the star's absolute magnitude. The program calculates the star's luminosity in solar luminosities (L). Enter the star's luminosity value above. 
    \item \textbf{\textit{With both the Temperature and Luminosity sliders}} set to the values you just found, the program calculates the star's radius in solar radii (R)-in other words, how many times larger or smaller than the star is. Enter the value in the table. This tells you how many times bigger the star's diameter is than the Sun's diameter. 
    \item Which region does your star fall closest to in the H-R diagram?
\end{enumerate}

\subsection*{3. The Stars We See}
In \textit{Stellarium} turn on constellation boundaries (B) and labels (V). Each table will be assigned a constellation: : 1 Boötes, 2 Canis Major, 3 Cassiopeia, 4 Centaurus, 5 Cygnus, 6 Lyra, 7 Orion, 8 Sagittarius, 9 Scorpius, 10 Ursa Minor, 11 Virgo. \begin{enumerate} [a.]
    \item Find the temperature, radii, and other properties of as many of the brightest stars in your constellation as there are people at your table-break up the work between everyone at the table by writing the names of the stars on your Table worksheet and assigning them to your group members. Write the information for your own star below, and copy this into the Table Worksheet after you have finished. 
    \item On the whiteboard corresponding to the distance of your star, draw a circle to represent your star. The circle should have a \textbf{diameter} in centimeterse equal to the size factor of your star. Use a marker color matching the following scale depending on the spectral type: O, B, A – blue; F, G – green; K, M – red. Write the spectral type of your star next to the star.
\end{enumerate}

\subsection*{4. Nearby Stars}
For this activity you are going to examine nearby stars in your constellation. To find them go to Wikipedia's List of Stars in [Your Constellation] or follow the link in Moodle. Sort the stars in order of distance, and look at the nearest stars, one for each person at your table. 

In the H-R Diagram Explorer, change the X-Axis Scale to "spectral type." Now when you adjust the Temperature slider you should get the arrow along the x-axis positioned in your star's spectral type (O,B,A,F,G,K,M) and the arabic numeral (0-9) following that letter indicates where in the range the star lies. For example, the Sun, a G2 star, is toward the left (hotter) side of the range of G-type stars, G0 would be at the left edge, G9 at the right edge.
\begin{enumerate} [a.]
    \item Enter the information for the closest star found at your computer below and copy the information into the Table Worksheet, and compare your results at your table.
    \item In the H-R Diagram Explorer web app, and click on "the nearest stars" and "the brightest stars." Why do these look so different?
    \item How do the stars you can see by eye at night compare to the Sun?
    \item How do the nearby stars compare to the Sun?
    \item Is the Sun a "typical" star
\end{enumerate}

\subsection*{5. Lab Quiz on Moodle}
Go to the Lab 9 section on the Moodle page and complete the End-of-Lab Quiz. Write your name on the Table Worksheet and hand it in.

To learn more about the H-R diagram, visit https://sci.esa.int/gaia-stellar-family-portrait/

\end{document}