\documentclass[main.tex]{subfiles}

\begin{document}
\subsubsection{Stellar Properties}
\begin{enumerate}
\item In groups of up to 3, open up Stellarium and turn off atmosphere (A) and ground (G) visualization.
\item Find the star Procyon. From the information panel, find:
	\begin{enumerate}
	\item Absolute magnitude
	\item Color index (B-V)
	\item Distance
	\item Spectral type
	\end{enumerate}
\end{enumerate}
N.B.: It's a lot easier to find this info in Stellarium if you Configure (F2) the Information to show just what you need. Open up the configuration window (F2), then click the information tab and click ``None" at the top. Select the following to display: Name, Catalog number, Visual magnitude, Absolute magnitude, Distance, and Additional information. (If you use the web app, you will need to look up Procyon on Wikipedia and look in the information panel on the right side of its page.)

\begin{enumerate}[a.]
\item What is the star's magnitude? What is its absolute magnitude? Why do these differ? 

\rule{15cm}{.15mm}
\item Fill in the first 5 columns below using the information in Stellarium:
\end{enumerate}

\subsubsection{Stellar Sizes}
\begin{table}[htbp]
%\caption{default}
\begin{center}
\begin{tabular}{|p{1.5cm}|p{1.5cm}|p{1cm}|p{1.5cm}|p{1.5cm}|p{1.5cm}|p{1.5cm}|p{1.5cm}|p{1.5cm}|}\hline
Name & Absolute Magnitude & B-V & Distance (light-years) & Spectral Type & Tempe-rature (K) & Lumino-sity ($\times$ solar) & Size ($\times$ solar) & Location in HRD \\\hline
Procyon

 &&&&&&&&\\\hline
(bright)


 &&&&&&&&\\\hline
(close)


 &&&&&&&&\\\hline
\end{tabular}
\end{center}
\label{tab:star}
\end{table}%
We'll explore stellar properties using an interactive Hertzsprung-Russell (HR) diagram.
\begin{enumerate}
\item Start up a web browser, and go to this website (or click on the link in Moodle): \url{http://astronomy.nmsu.edu/geas/labs/hrde/hrd_explorer.html}
\item Under Plot Labels:
	\begin{enumerate}
	\item Set the x-axis scale to ``B-V color index"
	\item Set the y-axis scale to ``magnitude"
	\end{enumerate}
\item Click on the HR diagram where Procyon is located by its absolute magnitude and B-V value.
\item Adjust the \textbf{Temperature} slider so that the red arrow on the bottom axis of the graph matches the star's B-V color value. Enter the star's \textbf{temperature} in the table.
\item Adjust the \textbf{Luminosity} slider so the red arrow on the left axis of the graph matches the star's absolute magnitude. The program calculates the star's luminosity in solar luminosities (L). Enter the star's \textbf{luminosity} value in the table. 
\item With the Temperature and Luminosity sliders set to the values you just found, the program calculates the star's radius in solar radii (R), in other words, how many times bigger the star's diameter is than the Sun's diameter. Enter the star's \textbf{size} value in the table.
\item Turn on the Main sequence and luminosity classes by clicking them.
\item Which region is Procyon located closest to: Main Sequence line (MS), red giants (RG), supergiants (SG), or white dwarfs (WD)?
\end{enumerate}

\subsubsection{The Stars We See}
Let's examine a sample of bright stars.

\begin{table}[htbp]
\caption{Colour code for spectral types}
\begin{center}
\begin{tabular}{|c|c|c|c|c|c|}\hline
O/B & A & F & G & K & M \\\hline
Violet & Blue & Green & Yellow & Orange & Red \\\hline
\end{tabular}
\end{center}
\label{tab:col}
\end{table}

\begin{enumerate}
\item Return to Stellarium and turn on constellation labels (V) and boundaries (B).
\item In your assigned constellation, find the bright stars and pick one each. Then, find their properties and fill them in the table as before. (If you are using the web app, use Wikipedia.)
\item Fill in the properties of your star on your table worksheet too.
\item Time to \emph{draw} on your art talents! On a blank piece of paper, draw circles representing your stars where:
	\begin{enumerate}
	\item the diameter corresponds to its size (1 cm for the Sun)
	\item the colour corresponds to its spectral type according to Table~\ref{tab:col}
	\item they are labelled with their name and spectral type.
	\end{enumerate}
\item On the whiteboard corresponding to the distance of your star, draw a circle to represent your star similar as above. Write the spectral type of your star next to it.
\end{enumerate}

\subsubsection{Nearby Stars}
Now we'll look at closest stars instead of brightest stars.
\begin{enumerate}
\item Go to the Wikipedia page ``List of stars in [your constellation]".
\item Sort by distance by clicking the up/down arrows in the Distance column.
\item Pick a different nearest star each and find its properties.
\item On the HR Diagram Explorer page, change the X-Axis Scale to ``Spectral type".
\item As before, adjust the Temperature slider so that the red arrow on the bottom axis of the graph matches the star's B-V color value. The red arrow should now also point to your star's spectral type (O,B,A,F,G,K,M). The arabic numeral (0-9) following that letter indicates where in the range the star lies, with 0 towards the left (hotter) edge and 9 towards the right (cooler) edge.

The Sun is a G2 star. Where on the diagram would it be located?
\item Copy your answers into your table and the table worksheet. Compare your answers.
\end{enumerate}

Answer the following questions.
\begin{enumerate}
\item In the H-R Diagram Explorer web app, and click on "the nearest stars" and "the brightest stars." Why do these look so different?

\rule{15cm}{.15mm}
\item How do the stars you can see by eye at night compare to the Sun?

\rule{15cm}{.15mm}
\item How do the nearby stars compare to the Sun?

\rule{15cm}{.15mm}
\item Is the Sun a "typical" star?

\rule{15cm}{.15mm}
\end{enumerate}

\subsubsection{Moodle Lab Quiz}
Go to the section for this lab on the Moodle page and complete the End-of-Lab Quiz. Write your name on the Table Worksheet and hand it in.

To learn more about the H-R diagram, visit \url{https://sci.esa.int/gaia-stellar-family-portrait/}

\end{document}